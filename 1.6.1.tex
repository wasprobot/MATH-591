\documentclass{article}

\usepackage{amsmath}
\usepackage{amssymb}

\begin{document}

\title{1.6.1 - Structures}
\date{\today}

\maketitle

\section*{1}
Consider the structure constructed in Example 1.6.2. 
Find the value of each of the following: 
\begin{itemize}
    \item $0+0=
        +(\text{Bottom},\text{Bottom})=
        \text{Oberon}$

    \item $0E0=
        E(\text{Bottom},\text{Bottom})=
        \text{Puck}$

    \item $S0\cdot SS0=
        \cdot(S(\text{Bottom}), S(S(\text{Bottom})))=
        \cdot(\text{Titania}, \text{Bottom})=
        \text{Titania}
        $
\end{itemize}

Do you think $0<0$ in this structure?
$$
    0<0=
    <(\text{Bottom},\text{Bottom})=\text{No}
$$

% \section*{2}
% Suppose that $\mathcal{L}$ is the language ${0,+,<}$. 
% Let's work together to describe an $\mathcal{L}$-structure 
% $\mathfrak{A}$. Let the universe $A$ be the set consisting 
% of all of the natural numbers together with Ingrid Bergman and Humphrey Bogart.
% You decide on the interpretations of the symbols. 

% \begin{itemize}
%     \item What is the value of 5 + Ingrid?
%     \item Is Bogie < 0? 
% \end{itemize}

\section*{3} 
Here is a language consisting of one constant symbol, 
one 3-ary function symbol, and one binary relation symbol: 
$\mathcal{L}$ is $\{\flat,\sharp,\natural\}$. 

\begin{itemize}
    \item Describe an $\mathcal{L}$-model that has as its universe $\mathbb{R}$, 
    the set of real numbers.
    $$
        \mathfrak{A} = (\mathbb{R}, \flat^\mathfrak{A}, \sharp^\mathfrak{A}, \natural^\mathfrak{A})
    $$

    \item Describe another $\mathcal{L}$-model that has a finite universe.
    $$
        \mathfrak{A} = (\{1,2,3\}, \flat^\mathfrak{A}, \sharp^\mathfrak{A}, \natural^\mathfrak{A})
    $$

\end{itemize}

\section*{4}
A short paragraph explaining the difference between a language
and a structure for a language.

A language is a set of symbols, and a structure is an interpretation of those symbols.
A structure is a set of objects, which have meaning. For example,
the language $\mathcal{L} = \{0,+,\cdot,S\}$ has the structure $\mathfrak{A} = (\mathbb{N}, 0^\mathfrak{A}, +^\mathfrak{A}, \cdot^\mathfrak{A}, S^\mathfrak{A})$,
where $0^\mathfrak{A}$ is the number zero, $+^\mathfrak{A}$ is the addition function, and $\cdot^\mathfrak{A}$ is the multiplication function
and $S^\mathfrak{A}$ is the unary successor function.
Without a structure, the symbols in a language have no meaning.

\section*{6}
Let $\mathcal{L}$ be $\{0, f, g, R\}$, where $0$ is a constant symbol, 
$f$ is a unary function symbol, $g$ is a binary function symbol, 
and $R$ is a $3$-ary relation symbol.\\

Using $\mathcal{C}$, the set of all $\mathcal{L}$-terms, 
we define a set 
$$
    \mathcal{C'} = (\forall t \in \mathcal{C})(\lnot{t})
$$

The desired $\mathcal{L}$-structure can be defined as:
$$
    \mathfrak{C} = \{ \mathcal{C'}, 0^\mathfrak{C}, f^\mathfrak{C}, g^\mathfrak{C}, R^\mathfrak{C} \}
$$.

\end{document}
